\chapter{Planung}
\section{Projektziele}
\subsection{MUSS-Ziele}

\subsubsection{Netzwerk aufbauen}
Eines unserer Ziele ist es, ein funktionierendes und erweiterbares Netzwerk aufzubauen. Benutzerprofile, Netzlaufwerke und ein FTP-Zugang werden am Server eingerichtet, um die Kommunikation der Mitarbeiter untereinander zu erleichtern. Somit k\"onnen die Mitarbeiter ihre Dateien zentral abspeichern und miteinander teilen.

\subsubsection{Installation und Inbetriebnahme eines Lagerprogramms und eines Rechnungsmanager, sowie die Einschulung der Mitarbeiter der Firma Skoda}
Damit kein Warenbestand verloren gehen kann (Schwund) und es eine automatische Kontrolle \"uber den aktuellen Lagerbestand gibt (Zufl\"usse und Abfl\"usse), werden wir das aktuelle Lagerverwaltungsprogramm verbessern. Am Ende unseres Projekts wird das Unternehmen nicht wie bisher alle Rechnungen mit Excel erstellen m\"ussen, sondern kann auf ein Programm mit einer Datenbank zur\"uckgreifen. Das erleichtert die Arbeit und Kontrolle der Buchhaltung. Das Rechnungsprogramm soll mit der Datenbank des Lagerprogramms auf den Bestand zugreifen k\"onnen.
Damit die Mitarbeiter mit den neuen Programmen arbeiten k\"onnen, soll ein Benutzerhandbuch erstellt werden, welches in einer Einschulung den Mitarbeitern erkl\"art und vorgestellt wird.

\subsubsection{Hotspot}
Zu Werbezwecken wird f\"ur die Firma ein gratis Hotspot eingerichtet. Beim Verbinden mit dem Hotspot \"offnet sich als Startseite die aktuelle Firmenseite. 


\subsection{Optionale-Ziele (Soll-, Kann-Ziele)}

\subsubsection{Erstellung einer Webseite, Soziale Medien pflegen und verbessern}
Damit diese Firma \"uberall in Shkodra und auch ausserhalb bekannt wird, muss sie eine gute Internet-Werbung haben. Deshalb werden wir f\"ur das Unternehmen eine Webseite in Wordpress erstellen und auch die sozialen Netze bewerben (Facebook, Instagram, Twitter, Google, etc.).

\subsubsection{Installation und Inbetriebnahme eines Bilanzprogramms, sowie die Einschulung der Mitarbeiter der Firma Skoda}
Als Erweiterung zum Rechnugsmanager, soll ein Bilanzprogramm installiert werden, welches jeden Monat die Einnahmen und Ausgaben \"ubersichtlich darstellt.

\section{Projektplanung}
Um ein erfolgreiches Resultat zu erzielen, war eine gut \"uberlegte Planung wichtig. Zum Punkt Projektplanung geh\"oren die Treffen mit unserem Hauptbetreuer Herrn Philip Michel und die Besprechungen mit den Projektauftraggeber Bledi Llazari. Der Projektauftraggeber war mit unserem Projekt einverstanden, also durften wir Informationen \"uber die Firma bekommen. Nachdem wir die Informationen \"uber die Firma bekommen hatten, konnten wir die Formulare der Planung definieren und ausf\"ullen. Die Plannungspunkte sind in der Tabelle \ref{table:todoliste} zu sehen. 
\begin{table}[ht]
\caption{To Do Liste}
\centering
\begin{tabular}{l l l}
\hline\hline
Code & Bezeichung & Bis wann?  \\ [0.5ex]
\hline
1 & Recherche \\
1.1 & Informationen sammeln & 20.09.15 \\
1.2 & Materialien einkaufen & 28.09.15 \\
2 & Netzwerk \\
2.1 & Netzwerktopologie planen & 26.09.15 \\
2.2 & Netzwerk aufbauen & 02.10.15 \\
2.3 & Netzwerk konfigurieren & 11.10.15 \\
2.4 & Netzwerk testen & 15.10.15 \\
3 & Lager-Programm \\
3.1 & Lager-Programm analysieren & 05.11.15 \\
3.1.2 & Vorhandenes Programm bewerten & 15.11.15 \\
3.1.3 & Alternative auflisten & 25.11.15 \\
3.1.4 & Lager-Programm testen & 15.12.15 \\
3.2 & Lager-Programm Inbetriebnahme & 05.01.16 \\
3.3 & Erstellung des Benutzerhandbuches & 10.01.16 \\
4 & Bilanz-Programm \\
4.1 & Bilanz-Programm analysieren & 05.11.15 \\
4.1.2 & Vorhandenes Programm bewerten & 15.11.15 \\
4.1.3 & Alternative auflisten & 25.11.15 \\
4.1.4 & Bilanz-Programm testen & 15.12.15 \\
4.2 & Bilanz-Programm Inbetriebnahme & 05.01.16 \\
4.3 & Erstellung des Benutzerhandbuches & 10.01.16 \\
5 & Rechnungsverwaltungs-Programm \\
5.1 & Rechnungsverwaltungs-Programm analysieren & 05.11.15 \\
5.1.2 & Vorhandenes Programm bewerten & 15.11.15 \\
5.1.3 & Alternative auflisten & 25.11.15 \\
5.1.4 & Rechnungsverwaltungs-Programm testen & 15.12.15 \\
5.2 & Rechnungsverwaltungs-Programm  \\
5.3 & Erstellung des Benutzerhandbuches & 10.01.16 \\
6 & Webseite \\
6.1 & Domain kaufen & 11.01.16 \\
6.2 & Webseite planen & 14.01.16 \\
6.3 & Webseite erstellen & 25.01.16 \\
6.4 & Webseite designen & 01.02.16 \\
6.5 & Webseite testen & 03.02.16 \\
6.6 & Webseite ver\"offentlichen & 04.02.16 \\
7 & Abnahme \\
7.1 & Die Produkte dem Auftraggeber \"ubergeben & 05.02.16 \\
7.2 & Skriptum f\"ur die Mitarbeiterschulung machen. \\
8 & Dokumentation \\
8.1 & Statuspr\"asentation halten & 25.01.15 \\
8.2 & Diplomarbeit vorbereiten & 01.02.16 \\
8.3 & Diplomarbeit schreiben & 07.02.16 \\
8.4 & Endkorrektur der Diplomarbeit & 16.02.16 \\
8.5 & Diplomarbeit drucken & 25.03.16  \\
8.6 & Endpr\"asentation machen & 04.16 \\

\hline
\end{tabular}
\label{table:todoliste} 
\end{table}

\section{Team Vorstellung}
\begin{table}[ht]
\caption{Team Vorstellung}
\hspace*{-2cm}
\begin{tabular}{l l}
\hline\hline
Name & Aufgabe \\ [0.5ex]
\hline
    		   & Netzwerk \\
Franc Bushati  & Hotspot-Zone \\
    		   & Installation von Lagerprogramm, Bilanzprogramm und Rechnungsverwaltungsprogramm \\
\hline
               & Auswahl/Installation von Lagerprogramm, Bilanzprogramm und  \\
Igli Kadija	   & Rechnungsverwaltungsprogramm \\
    		   & Netzwerk \\
\hline    		   
               & CMS \\
Alfons Lazri   & Pflege der sozialen Medien \\
    		   & Netzwerk \\
\hline    		   
			   & Auswahl von Lagerprogramm, Bilanzprogramm und Rechnungsverwaltungsprogramm \\
Mergim Smajlaj & Dokumentation \\
    		   & Pflege der sozialen Medien \\
\hline    		   
      	       & Hotspot-Zone \\
Riad Kavaja	   & CMS \\	
    		   & Netzwerk \\
\hline
\end{tabular}
\label{table:teamvorstellung} 
\end{table}


\section{Projektmanagementmethode}
Wasserfallmethode
Als Projektmanagementmethode haben wir uns f\"ur das Wasserfallmodell entschieden. Dieses Modell pa{\ss}te f\"ur uns am besten und spiegelt sich in der Plannungspunkte, welche der Tabelle \ref{table:todoliste} zu entnehmen sind.